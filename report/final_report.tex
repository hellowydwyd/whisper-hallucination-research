% !TEX program = xelatex
% Whisper ASR 幻觉问题研究与缓解方案
% 中国传媒大学 智能音频处理 课程结课作业
% 
% 编译方式: xelatex 论文模板.tex
% 需要安装完整的 TeX Live 或 MiKTeX

\documentclass[12pt, a4paper]{article}

% ==================== 中文支持 ====================
\usepackage[UTF8]{ctex}
\usepackage{fontspec}

% ==================== 字体设置 ====================
% 根据 PDF 模板提取的字体信息配置

% 正文字体 - 宋体 12pt
\setCJKmainfont{SimSun}[AutoFakeBold=2, AutoFakeSlant=0.2]

% 黑体 - 用于封面题目
\setCJKsansfont{SimHei}[AutoFakeBold=2]

% 仿宋 - ctex 已预定义 \fangsong,无需重复定义

% 等线字体 - 用于标题
% 注意:如果系统没有等线字体,可以用黑体替代
\newCJKfontfamily\dengxian{DengXian}[
    BoldFont=DengXian Bold,
    AutoFakeBold=2
]
\newCJKfontfamily\dengxianlight{DengXian Light}
\newCJKfontfamily\dengxianbold{DengXian Bold}

% 英文字体
\setmainfont{Times New Roman}
\setsansfont{Arial}
\setmonofont{Consolas}

% ==================== 页面设置 ====================
\usepackage[
    a4paper,
    top=2.5cm,
    bottom=2.5cm,
    left=2.5cm,
    right=2.5cm,
    headheight=14pt
]{geometry}

% ==================== 页眉页脚 ====================
\usepackage{fancyhdr}
\usepackage{lastpage}

\pagestyle{fancy}
\fancyhf{}

% 页眉 - 宋体 10.45pt (约等于小五号)
\fancyhead[C]{\fontsize{10.45pt}{12pt}\selectfont\songti 智能音频处理}

% 页眉下划线
\renewcommand{\headrulewidth}{0.4pt}

% 页码 (如需要可取消注释)
% \fancyfoot[C]{\thepage}

% ==================== 标题格式 ====================
\usepackage{titlesec}

% 论文大标题 - 等线粗体 26pt
\newcommand{\papertitle}[1]{
    \begin{center}
        {\fontsize{26pt}{32pt}\selectfont\dengxianbold #1}
    \end{center}
    \vspace{1em}
}

% 一级标题 - 等线细体 16pt
\titleformat{\section}
    {\fontsize{15.95pt}{20pt}\selectfont\dengxianlight}
    {\thesection}{1em}{}

% 二级标题 - 等线 14pt  
\titleformat{\subsection}
    {\fontsize{14.05pt}{18pt}\selectfont\dengxian}
    {\thesubsection}{1em}{}

% 三级标题 - 等线细体 14pt
\titleformat{\subsubsection}
    {\fontsize{14.05pt}{18pt}\selectfont\dengxianlight}
    {\thesubsubsection}{1em}{}

% ==================== 其他设置 ====================
\usepackage{graphicx}       % 图片
\usepackage{booktabs}       % 表格美化
\usepackage{array}          % 表格增强
\usepackage{multirow}       % 表格合并
\usepackage{amsmath}        % 数学公式
\usepackage{amssymb}        % 数学符号
\usepackage{hyperref}       % 超链接
\usepackage{caption}        % 图表标题
\usepackage{subcaption}     % 子图
\usepackage{float}          % 浮动体控制
\usepackage{enumitem}       % 列表格式

% 图表标题格式 - 等线粗体 10.45pt
\captionsetup{
    font={small},
    labelfont={bf},
    labelsep=space
}

% 超链接样式
\hypersetup{
    colorlinks=true,
    linkcolor=black,
    citecolor=black,
    urlcolor=blue
}

% 行距
\linespread{1.5}

% 段落缩进
\setlength{\parindent}{2em}

% ==================== 文档开始 ====================
\begin{document}

% ==================== 正文 ====================

\section{研究任务介绍(Introduction)}

\subsection{任务描述}

本研究聚焦于 OpenAI Whisper 自动语音识别(ASR)模型的幻觉(Hallucination)问题。幻觉是指 ASR 模型在处理非语音音频(如静音、噪声、环境声音)时,生成与原始音频没有语音或语义联系的文本输出的现象。

本研究的主要任务包括四个方面:首先,复现并验证 Whisper 模型在非语音音频上的幻觉现象;其次,量化分析不同类型音频的幻觉率及幻觉内容分布;第三,对比评估多种幻觉缓解方案的有效性;最后,验证模型在正常语音识别任务上的准确性。

\subsection{研究意义}

Whisper 作为目前最先进的开源语音识别模型之一,在工业界和学术界有着广泛应用。然而,其幻觉问题可能带来严重后果。在可靠性方面,医疗、法律等关键领域的错误转录可能导致严重后果;在安全性方面,研究表明部分幻觉可能包含暴力、色情等有害内容;在用户体验方面,实时转录场景中的无意义输出会严重影响使用效果。因此,深入研究 Whisper 的幻觉问题并探索有效的缓解方案具有重要的理论和实践价值。

\subsection{国内外研究现状}

\textbf{Whisper 模型。}Radford 等人 [1] 于 2022 年提出 Whisper 模型,使用 680,000 小时的多语言弱监督数据进行训练,采用 Encoder-Decoder Transformer 架构,支持语音识别、翻译、语言识别等多任务。该模型在 LibriSpeech test-clean 数据集 [6] 上达到 2.5\% 的词错误率(WER),接近人类水平。

\textbf{幻觉问题研究。}Barański 等人 [3] 在 ICASSP 2025 会议上发表了对 Whisper 幻觉现象的大规模调查研究。通过对 301,317 个非语音音频文件的测试,发现幻觉发生率高达 40.3\%,其中 67\% 的幻觉来自 1,270 个重复短语,最常见的是 "thank you"(24.76\%)和 "thanks for watching"(10.32\%)等 YouTube 字幕相关内容。

\textbf{Calm-Whisper 解决方案。}Wang 等人 [2] 提出 Calm-Whisper 方法,通过分析发现 Whisper 解码器中的特定注意力头(\#1, \#6, \#11)是导致幻觉的主要原因。通过对这些 "crazy heads" 进行定向微调,将幻觉率从 99.97\% 降低至 15.51\%,同时保持 WER 仅增加约 0.07\%。

% ==================== 第二章 ====================
\section{研究内容和技术路线}

\subsection{研究内容}

本研究围绕以下四个核心问题展开:

\textbf{问题一:合成音频幻觉测试。}本研究首先测试 Whisper 在静音、白噪声、粉红噪声等合成音频上的幻觉表现,分析不同音频长度对幻觉率的影响,并研究 Whisper 参数(如 no\_speech\_threshold)对幻觉的影响。

\textbf{问题二:真实环境声音幻觉测试。}使用 ESC-50 环境声音数据集 [5] 进行大规模测试,分析不同声音类别(动物、自然、城市等)的幻觉率差异,并统计幻觉内容的分布特征。

\textbf{问题三:语音识别准确性验证。}使用 LibriSpeech test-clean 数据集 [6] 测试模型的词错误率(WER),验证模型在正常语音识别任务上的可靠性。

\textbf{问题四:缓解方案对比评估。}实现并测试 VAD(语音活动检测)预处理方案和 BoH(幻觉词袋)后处理方案,对比各方案单独使用和组合使用的效果。

\subsection{技术路线}

\subsubsection{实验框架设计}

本研究采用模块化的实验框架,包括五个核心组件。Whisper 模型封装模块对 OpenAI Whisper 模型进行封装,支持不同参数配置的转录。音频处理模块提供音频加载、生成、预处理等功能。评估指标模块实现 WER、CER、幻觉率、循环率等指标计算。VAD 预处理模块基于能量检测进行语音活动检测。BoH 后处理模块使用 Aho-Corasick 算法进行幻觉短语匹配和过滤。

\subsubsection{评估指标}

\textbf{词错误率(WER)}:
\begin{equation}
    \text{WER} = \frac{S + D + I}{N} \times 100\%
\end{equation}

其中 $S$ 为替换词数,$D$ 为删除词数,$I$ 为插入词数,$N$ 为参考文本总词数。

\textbf{幻觉率}:
\begin{equation}
    \text{Hallucination Rate} = \frac{\text{产生幻觉的样本数}}{\text{总样本数}} \times 100\%
\end{equation}

% ==================== 第三章 ====================
\section{实验结果及分析(重点内容)}

\subsection{实验结果}

\subsubsection{开发环境介绍}

硬件环境方面,本实验使用 NVIDIA GeForce RTX 系列 GPU 进行加速计算,系统内存为 16GB 以上。软件环境方面,操作系统为 Windows 10/11,编程语言为 Python 3.9,深度学习框架为 PyTorch 2.0+,语音识别模型使用 openai-whisper 的 large-v3 版本。

数据集方面,本研究使用两个公开数据集:ESC-50 环境声音数据集 [5] 包含 2000 个样本,涵盖 50 个类别,每类 40 个样本;LibriSpeech test-clean 数据集 [6] 包含 2620 个英语语音样本,用于评估模型的语音识别准确性。

\subsubsection{性能评估指标介绍}

\begin{table}[H]
\centering
\caption{性能评估指标}
\begin{tabular}{lll}
\toprule
\textbf{指标} & \textbf{定义} & \textbf{用途} \\
\midrule
WER & 词错误率 & 评估语音识别准确性 \\
CER & 字符错误率 & 评估字符级准确性 \\
幻觉率 & 产生幻觉的样本比例 & 评估幻觉严重程度 \\
循环率 & 存在重复输出的样本比例 & 评估循环幻觉 \\
\bottomrule
\end{tabular}
\end{table}

\subsubsection{实验结果和分析讨论}

\textbf{实验一:合成音频幻觉测试}

\begin{table}[H]
\centering
\caption{合成音频幻觉率}
\begin{tabular}{lcccc}
\toprule
\textbf{音频类型} & \textbf{样本数} & \textbf{幻觉率} & \textbf{循环率} & \textbf{主要幻觉内容} \\
\midrule
静音 & 25 & 100\% & 0\% & "Thank you." \\
白噪声 & 15 & 100\% & 0\% & "Thank you." \\
粉红噪声 & 15 & 100\% & 0\% & "Thank you." \\
\midrule
\textbf{总计} & \textbf{55} & \textbf{100\%} & \textbf{0\%} & - \\
\bottomrule
\end{tabular}
\end{table}

% 图1 - 请在此处插入 experiment_results.png
\begin{figure}[H]
\centering
\includegraphics[width=0.9\textwidth]{../output/experiment_results.png}
\caption{合成音频幻觉率分布}
\end{figure}

\textbf{分析}:Whisper large-v3 在所有合成非语音音频上都产生了幻觉,幻觉率达到 100\%。这一结果表明,当输入音频不包含任何语音信息时,模型仍会强制生成文本输出,而非保持沉默。最常见的幻觉输出是 "Thank you.",这与 Barański 等人 [3] 的研究结论一致。从技术角度分析,这种现象源于 Whisper 训练数据中包含大量 YouTube 视频字幕,其中视频结尾常见的致谢语句被模型过度拟合。此外,实验发现音频时长(1-30秒)对幻觉率没有显著影响,表明幻觉问题是模型架构层面的固有缺陷,而非特定输入条件触发。

\textbf{实验二:ESC-50 环境声音幻觉测试}

\begin{table}[H]
\centering
\caption{ESC-50 各类别幻觉率}
\begin{tabular}{lccc}
\toprule
\textbf{声音类别} & \textbf{样本数} & \textbf{幻觉数} & \textbf{幻觉率} \\
\midrule
自然声音 (natural) & 400 & 316 & 79.0\% \\
室内声音 (interior) & 400 & 271 & 67.8\% \\
城市声音 (exterior) & 400 & 264 & 66.0\% \\
人类非语音 (human\_non\_speech) & 400 & 223 & 55.8\% \\
动物声音 (animals) & 400 & 176 & 44.0\% \\
\midrule
\textbf{总计} & \textbf{2000} & \textbf{1250} & \textbf{62.5\%} \\
\bottomrule
\end{tabular}
\end{table}

% 图2 - 请在此处插入 esc50_results.png
\begin{figure}[H]
\centering
\includegraphics[width=0.9\textwidth]{../output/esc50_results_20260114_053915.png}
\caption{ESC-50 各类别幻觉率对比}
\end{figure}

\begin{table}[H]
\centering
\caption{幻觉内容分布}
\begin{tabular}{lcc}
\toprule
\textbf{幻觉内容} & \textbf{出现次数} & \textbf{占比} \\
\midrule
"Thank you." & 617 & 49.4\% \\
"Thanks for watching!" & 134 & 10.7\% \\
"." & 68 & 5.4\% \\
"Oh" & 23 & 1.8\% \\
"you" & 20 & 1.6\% \\
其他 & 388 & 31.0\% \\
\bottomrule
\end{tabular}
\end{table}

\textbf{分析}:ESC-50 数据集的实验结果揭示了 Whisper 幻觉问题的类别差异性。自然声音(如雨声、风声、雷声)的幻觉率最高(79\%),这可能是因为这类声音的频谱特征与语音信号差异较大,模型更容易产生误判。相比之下,动物声音的幻觉率最低(44\%),推测原因是部分动物叫声(如狗吠、猫叫)在频率和节奏上与人类语音有一定相似性,可能被模型识别为非语音而抑制输出。人类非语音类别(如咳嗽、打喷嚏、笑声)的幻觉率为 55.8\%,处于中等水平,说明即使是人类发出的非语言声音,模型也无法有效区分并抑制幻觉。从幻觉内容分布来看,"Thank you." 和 "Thanks for watching!" 两个短语合计占比超过 60\%,高度集中的分布特征为基于词袋的后处理方法提供了理论依据。

\textbf{实验三:LibriSpeech WER 测试}

\begin{table}[H]
\centering
\caption{LibriSpeech 语音识别性能}
\begin{tabular}{lc}
\toprule
\textbf{指标} & \textbf{数值} \\
\midrule
测试样本数 & 2620 \\
总词数 & 52,576 \\
\textbf{总体 WER} & \textbf{3.71\%} \\
CER & 1.74\% \\
完美识别率 (WER=0) & 67.9\% \\
\bottomrule
\end{tabular}
\end{table}

% 图3 - 请在此处插入 librispeech_wer.png
\begin{figure}[H]
\centering
\includegraphics[width=0.9\textwidth]{../output/librispeech_wer_20260114_162920.png}
\caption{LibriSpeech WER 分布}
\end{figure}

\textbf{分析}:LibriSpeech test-clean 数据集的实验结果验证了 Whisper large-v3 在正常语音识别任务上的可靠性。3.71\% 的总体 WER 与 Radford 等人 [1] 原论文报告的 2.5\% 存在约 1.2 个百分点的差距,这一差异可能源于以下因素:本实验使用的解码参数与原论文略有不同;实验环境(Windows + CUDA)与原论文的训练/测试环境存在差异。尽管如此,67.9\% 的样本达到完美识别(WER=0)的结果表明,Whisper 在处理清晰语音时具有优异的识别能力。这一发现具有重要意义:它证明了幻觉问题主要发生在非语音输入场景,而非模型整体能力的缺陷。因此,针对性的前后处理方案可以有效缓解幻觉问题,同时不影响正常的语音识别功能。

\textbf{实验四:缓解方案对比}

\begin{table}[H]
\centering
\caption{各缓解方案效果对比}
\begin{tabular}{lccc}
\toprule
\textbf{方法} & \textbf{幻觉率} & \textbf{平均输出长度} & \textbf{幻觉降低幅度} \\
\midrule
原始 Whisper & 77.2\% & 9.2 字符 & (基准) \\
VAD 预处理 & 57.9\% & 7.2 字符 & -25.0\% \\
BoH 后处理 & 20.7\% & 2.3 字符 & -73.2\% \\
\textbf{VAD + BoH 组合} & \textbf{20.0\%} & \textbf{2.2 字符} & \textbf{-74.1\%} \\
\bottomrule
\end{tabular}
\end{table}

% 图4 - 请在此处插入 mitigation_comparison.png
\begin{figure}[H]
\centering
\includegraphics[width=0.9\textwidth]{../output/mitigation_comparison_20260114_190946.png}
\caption{缓解方案效果对比}
\end{figure}

\textbf{分析}:缓解方案对比实验揭示了不同方法的优劣特点。VAD 预处理方法将幻觉率从 77.2\% 降低至 57.9\%,降幅为 25.0\%。该方法通过检测音频中的语音活动区间,在非语音段直接返回空字符串,从而避免幻觉产生。然而,基于能量阈值的简单 VAD 算法对噪声较为敏感,当环境声音的能量超过阈值时仍会触发转录,导致效果有限。BoH 后处理方法表现更为出色,将幻觉率降低至 20.7\%,降幅达 73.2\%。该方法利用 Aho-Corasick 多模式匹配算法,在 O(n) 时间复杂度内检测输出文本中的已知幻觉短语并将其过滤。由于幻觉内容高度集中(前 3 种幻觉占比超过 65\%),BoH 方法能够有效捕获大部分幻觉输出。VAD + BoH 组合方案的幻觉率为 20.0\%,相比单独使用 BoH 仅提升 0.7 个百分点,说明两种方法存在功能重叠,BoH 已覆盖了大部分 VAD 能够处理的场景。从工程实践角度,BoH 方法具有更高的性价比:实现简单、计算开销低、效果显著,是目前最推荐的幻觉缓解方案。

\subsection{实验结果总结}

\textbf{(1) 实验结果分析}

\textbf{核心发现一:Whisper 在非语音音频上存在严重幻觉问题。}实验结果表明,在合成音频(静音、噪声)上幻觉率高达 100\%,在 ESC-50 真实环境声音数据集 [5] 上幻觉率为 62.5\%。从类别分布来看,自然声音类别的幻觉率最高,达到 79\%,而动物声音类别最低,为 44\%。

\textbf{核心发现二:幻觉内容高度集中。}统计分析显示,"Thank you." 占所有幻觉输出的 49.4\%,前 3 种幻觉内容的累计占比超过 65\%。这些内容明显来源于 YouTube 字幕训练数据 [3],反映了训练数据偏差对模型行为的影响。

\textbf{核心发现三:模型语音识别能力可靠。}在 LibriSpeech test-clean 数据集 [6] 上,模型的 WER 为 3.71\%,接近原论文 [1] 报告的水平,其中 67.9\% 的样本达到完美识别(WER=0)。这表明幻觉问题主要出现在非语音输入场景。

\textbf{核心发现四:BoH 后处理是最有效的缓解方案。}实验对比表明,基于 Barański 等人 [3] 提出的幻觉词袋(BoH)方法,可将幻觉率降低 74.1\%(从 77.2\% 降至 20.0\%),且实现简单,无需修改模型结构。

\textbf{(2) 问题分析及未来可能采取的方法改进}

\textbf{当前方案的局限性。}本研究的缓解方案存在以下局限:首先,VAD 方法较为简单,基于能量的检测对噪声和语音的区分能力有限,容易出现误判;其次,BoH 幻觉短语列表是预定义的静态列表,无法覆盖所有非典型幻觉内容;最后,本研究仅采用前后处理方法,未涉及 Calm-Whisper [2] 等模型微调方案。

\textbf{未来改进方向。}针对上述局限性,未来研究可从以下方向改进:集成深度学习 VAD 模型(如 Silero VAD)以提高语音检测准确性;建立动态更新机制,根据实际应用场景扩展幻觉短语列表;复现 Calm-Whisper [2] 方法,从模型注意力机制层面根本解决幻觉问题。

% ==================== 第四章 ====================
\section{结论}

本研究对 OpenAI Whisper large-v3 模型的幻觉问题进行了系统性的实验研究。

在幻觉问题验证方面,实验结果表明 Whisper 在合成音频上的幻觉率达到 100\%,在 ESC-50 环境声音数据集 [5] 上的幻觉率为 62.5\%,证实了该问题的严重性和普遍性。

在幻觉内容分析方面,约 50\% 的幻觉输出为 "Thank you.",这一高度集中的分布规律与模型训练数据中大量 YouTube 字幕内容有直接关系 [3],揭示了训练数据偏差对模型行为的深刻影响。

在语音识别能力验证方面,模型在 LibriSpeech test-clean 数据集 [6] 上达到 3.71\% 的词错误率,证明幻觉问题主要出现在非语音输入场景,而在正常语音识别任务上模型表现可靠。

在缓解方案评估方面,BoH 后处理方法 [3] 可将幻觉率降低 74.1\%,是一种实现简单且效果显著的解决方案,具有较高的实用价值。

本研究的实验结果与相关论文 [2][3] 的发现高度一致,为 Whisper 模型在实际应用中的幻觉防护提供了参考依据。

% ==================== 参考文献 ====================
\section*{参考文献}
\addcontentsline{toc}{section}{参考文献}

\begin{enumerate}[label={[\arabic*]}]
    \item Radford, A., Kim, J. W., Xu, T., et al. Robust Speech Recognition via Large-Scale Weak Supervision[C]. ICML, 2022.
    
    \item Wang, Y., et al. Calm-Whisper: Reduce Whisper Hallucination On Non-Speech By Calming Crazy Heads Down[C]. Interspeech, 2025.
    
    \item Barański, M., et al. Investigation of Whisper ASR Hallucinations Induced by Non-Speech Audio[C]. ICASSP, 2025.
    
    \item Koenecke, A., et al. Racial disparities in automated speech recognition[J]. PNAS, 2020.
    
    \item Piczak, K. J. ESC: Dataset for Environmental Sound Classification[C]. ACM MM, 2015.
    
    \item Panayotov, V., et al. LibriSpeech: An ASR corpus based on public domain audio books[C]. ICASSP, 2015.
\end{enumerate}

\end{document}
